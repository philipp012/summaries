\documentclass{article}
\usepackage{fancyhdr}
\usepackage{graphicx}
\usepackage[german]{babel}
\pagestyle{fancy}

\author{Philipp Kiss}

\lhead{Philipp Kiss}
\rhead{Analysis 2}

\begin{document}
\section{Integrationsregeln}
\subsection{Faktorregel}
Ein konstanter Faktor im Integranden kann vor das Integralzeichen gezogen werden.
$$ \int_{}^{} c \cdot f(x) \mathrm{d}x = c \cdot \int_{}^{} f(x) \mathrm{d}x $$
\subsection{Summen- und Differenzregel}
Summen und Differenzen im Integranden können auseinander gezogen werden und als zwei Integrale geschrieben werden.
$$ \int_{}^{} f(x) + g(x) - h(x) \mathrm{d}x = \int_{}^{} f(x) \mathrm{d}x + \int_{}^{} g(x) \mathrm{d}x - \int_{}^{} h(x) \mathrm{d}x $$

\subsection{Partielle Integration}
Die partielle Integration ist das Pendant zur Produktregel beim Ableiten und wird benötigt, wenn im Integranden Funktionen multipliziert werden. Die Formel der partiellen Integration ist
$$u(x) \cdot v(x) - \int_{}^{} u'(x) \cdot v(x) \mathrm{d}x $$
Wobei für eine Funktion $u(x)$ und für die andere $v'(x)$ eingesetzt wird.

Beispiel:
$$ \int_{}^{} x \cdot e^{x} \mathrm{d}x $$
$$u(x) = x, v'(x) = e^{x}$$
Daraus folgt
$$x \cdot e^{x} - \int_{}^{} 1 \cdot e^{x} \mathrm{d}x = x \cdot e^{x} - e^{x} = e^{x}(x-1)$$

\subsection{Partialbruchzerlegung}
Mithilfe einer Partialbruchzerlegung kann man eine rationale Funktion als eine Addition von einfacheren Brüchen betrachten. Dies erlaubt uns die einfacheren Brüche einzeln zu integrieren und danach aufzusummieren. Dafür wird das Nennerpolynom als erstes in die Nullstellenform gebracht, woraufhin man die einzelnen Nullstellen ihrer Vielfachheitnach als seperate Brüche mit unbekannten Zählern darstellen kann.Die Summe von Brüchen kann mit dem ursprünglichen Nennerpolynom multipliziert und dem ursprünglichen Zählerpolynom gleichgesetzt werden. Das erlaubt uns ein Gleichungssystem aufzustellen, mit welchem wir die unbekannten Zähler der Partialbrüche berechnen können.Die Summe von Brüchen kann mit dem ursprünglichen Nennerpolynom multipliziert und dem ursprünglichen Zählerpolynom gleichgesetzt werden. Das erlaubt uns ein Gleichungssystem aufzustellen, mit welchem wir die unbekannten Zähler der Partialbrüche berechnen können. Schlussendlich werden die Partialbrüche einzeln integriert.

Beispiel:
$$ \int_{}^{} \frac{5x+11}{x^{2}+3x-10} \mathrm{d}x = \int_{}^{} \frac{5x+11}{(x+5)(x-2)}  \mathrm{d}x $$
$$= \int_{}^{} \frac{A}{x+5} + \frac{B}{x-2}  \mathrm{d}x \Rightarrow 5x+11 = A(x-2)+b(x+5)$$
$$x = -5 \Rightarrow A = 2$$
$$x=2 \Rightarrow B = 3$$
$$ \int_{}^{} \frac{2}{x+5} + \frac{3}{x-2}  \mathrm{d}x = \int_{}^{} 2 \frac{1}{x+5} + 3 \frac{1}{x-2}  \mathrm{d}x $$
$$=2 \ln |x+5|+ 3\ln|x-2|+c$$


\subsection{Substitution}

\end{document}
