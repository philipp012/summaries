\documentclass{article}
\usepackage{fancyhdr}
\usepackage{graphicx}
\usepackage[german]{babel}
\pagestyle{fancy}

\author{Philipp Kiss}

\lhead{Philipp Kiss}
\rhead{Analysis 2}

\begin{document}
\section{Integrationsregeln}
\subsection{Faktorregel}
Ein konstanter Faktor im Integranden kann vor das Integralzeichen gezogen werden.
$$ \int_{}^{} c \cdot f(x) \mathrm{d}x = c \cdot \int_{}^{} f(x) \mathrm{d}x $$
\subsection{Summen- und Differenzregel}
Summen und Differenzen im Integranden können auseinander gezogen werden und als zwei Integrale geschrieben werden.
$$ \int_{}^{} f(x) + g(x) - h(x) \mathrm{d}x = \int_{}^{} f(x) \mathrm{d}x + \int_{}^{} g(x) \mathrm{d}x - \int_{}^{} h(x) \mathrm{d}x $$

\subsection{Partielle Integration}
Die partielle Integration ist das Pendant zur Produktregel beim Ableiten und wird benötigt, wenn im Integranden Funktionen multipliziert werden. Die Formel der partiellen Integration ist
$$u(x) \cdot v(x) - \int_{}^{} u'(x) \cdot v(x) \mathrm{d}x $$
Wobei für eine Funktion $u(x)$ und für die andere $v'(x)$ eingesetzt wird.

Beispiel:
$$ \int_{}^{} x \cdot e^{x} \mathrm{d}x $$
$$u(x) = x, v'(x) = e^{x}$$
Daraus folgt
$$x \cdot e^{x} - \int_{}^{} 1 \cdot e^{x} \mathrm{d}x = x \cdot e^{x} - e^{x} = e^{x}(x-1)$$

\subsection{Substitution}

\end{document}
