\documentclass{article}
\usepackage{fancyhdr}
\usepackage{graphicx}
\usepackage[german]{babel}
\pagestyle{fancy}

\author{Philipp Kiss}

\lhead{Philipp Kiss}
\rhead{Analysis 2}

\begin{document}
\section{Integrationsregeln}
\subsection{Faktorregel}
Ein konstanter Faktor im Integranden kann vor das Integralzeichen gezogen werden.
$$ \int_{}^{} c \cdot f(x) \mathrm{d}x = c \cdot \int_{}^{} f(x) \mathrm{d}x $$
\subsection{Summen- und Differenzregel}
Summen und Differenzen im Integranden können auseinander gezogen werden und als zwei Integrale geschrieben werden.
$$ \int_{}^{} f(x) + g(x) - h(x) \mathrm{d}x = \int_{}^{} f(x) \mathrm{d}x + \int_{}^{} g(x) \mathrm{d}x - \int_{}^{} h(x) \mathrm{d}x $$

\subsection{Partielle Integration}
Die partielle Integration ist das Pendant zur Produktregel beim Ableiten und wird benötigt, wenn im Integranden Funktionen multipliziert werden. Die Formel der partiellen Integration ist
$$u(x) \cdot v(x) - \int_{}^{} u'(x) \cdot v(x) \mathrm{d}x $$
Wobei für eine Funktion $u(x)$ und für die andere $v'(x)$ eingesetzt wird.

Beispiel:
$$ \int_{}^{} x \cdot e^{x} \mathrm{d}x $$
$$u(x) = x, v'(x) = e^{x}$$
Daraus folgt
$$x \cdot e^{x} - \int_{}^{} 1 \cdot e^{x} \mathrm{d}x = x \cdot e^{x} - e^{x} = e^{x}(x-1)$$

\subsection{Partialbruchzerlegung}
Mithilfe einer Partialbruchzerlegung kann man eine rationale Funktion als eine Addition von einfacheren Brüchen betrachten. Dies erlaubt uns die einfacheren Brüche einzeln zu integrieren und danach aufzusummieren. Dafür wird das Nennerpolynom als erstes in die Nullstellenform gebracht, woraufhin man die einzelnen Nullstellen ihrer Vielfachheitnach als seperate Brüche mit unbekannten Zählern darstellen kann.Die Summe von Brüchen kann mit dem ursprünglichen Nennerpolynom multipliziert und dem ursprünglichen Zählerpolynom gleichgesetzt werden. Das erlaubt uns ein Gleichungssystem aufzustellen, mit welchem wir die unbekannten Zähler der Partialbrüche berechnen können.Die Summe von Brüchen kann mit dem ursprünglichen Nennerpolynom multipliziert und dem ursprünglichen Zählerpolynom gleichgesetzt werden. Das erlaubt uns ein Gleichungssystem aufzustellen, mit welchem wir die unbekannten Zähler der Partialbrüche berechnen können. Schlussendlich werden die Partialbrüche einzeln integriert.

Beispiel:
$$ \int_{}^{} \frac{5x+11}{x^{2}+3x-10} \mathrm{d}x = \int_{}^{} \frac{5x+11}{(x+5)(x-2)}  \mathrm{d}x $$
$$= \int_{}^{} \frac{A}{x+5} + \frac{B}{x-2}  \mathrm{d}x \Rightarrow 5x+11 = A(x-2)+b(x+5)$$
$$x = -5 \Rightarrow A = 2$$
$$x=2 \Rightarrow B = 3$$
$$ \int_{}^{} \frac{2}{x+5} + \frac{3}{x-2}  \mathrm{d}x = \int_{}^{} 2 \frac{1}{x+5} + 3 \frac{1}{x-2}  \mathrm{d}x $$
$$=2 \ln |x+5|+ 3\ln|x-2|+c$$


\subsection{Substitution}
Mithilfe einer Substitution können wir verschachtelte Funktionen einfacher integrieren. Dafür substituiert man eine Funktion, welche die Unbekannte beinhaltet, durch einen Term \textit{u}. Um diesen Term zu ersetzen, müssen, falls vorhanden, die Grenzen des Integrals ebenfalls zu ``\textit{u}-Grenzen'' umgeformt werden.

\section{Anwendungen der Integralrechnung}
\subsection{Mittelwert einer Funktion}
Der Mittelwert einer Funktion wird immer in Abhängigkeit von 2 Grenzen berechnet. $$\mu= \frac{1}{b-a} \int_{a}^{b} f(x) \mathrm{d}x  $$
\subsection{Rotationskörper}
\subsubsection{Volumen}
Mithilfe eines Integrals kann man das Volumen eines Körpers berechnen, welcher entsteht, wenn man einen Funktionsabschnitt um die X-Achse rotiert.
Die Formel dazu lautet
$$V= \pi \int_{a}^{b} f(x)^{2} \mathrm{d}x $$
\subsubsection{Mantelfläche}
Die Mantelfläche eines Rotationköpers kann durch die aufsummierte Fläche von Kegelstümpfen approximiert werden.
\newline
Notation als Summe:
$$M =  \lim_{n \to \infty} \pi \cdot \sum_{k=1}^{n} (f(x_k) + f(x_{k+1}) \cdot \sqrt[]{1+\left( \frac{\Delta_{y_k}}{\Delta_{x_k}}^{2} \right)} \cdot \Delta_{x_k}$$
Notation als Integral:
$$M = 2\pi \int_{a}^{b} f(x) \cdot \sqrt[]{1+(f'(x))^{2}} \mathrm{d}x $$
\subsubsection{Schwerpunkt}
Mithilfe des Volumens kann der Schwerpunkt eines Rotationkörpers ermittelt werden.
$$x_s = \frac{\pi}{V} \int_{a}^{b} x \cdot f(x)^{2} \mathrm{d}x $$
\subsection{Bogenlänge einer Kurve}
Die Bogenlänger einer Kurve kann mithilfe eines Integrals als Summe von infinitesimalen Geraden Strecken approximiert werden.
$$L = \int_{a}^{b} \sqrt[]{1+(f'(x)}^{2} \mathrm{d}x $$
\subsection{Ebene Flächen}
\subsubsection{Schwerpunkt}
Der Schwerpunkt einer ebenen Fläche die durch zwei Funktionen eingegrenzt wird kann ebenfalls mithilfe von integrieren approximiert werden.
$$x_s = \frac{1}{A} \int_{a}^{b} x \cdot (f(x)-g(x)) \mathrm{d}x $$
$$x_s = \frac{1}{2A} \int_{a}^{b} x \cdot (f(x)^{2}-g(x)^{2}) \mathrm{d}x $$
\subsubsection{Fläche}
Die Fläche, welche von zwei Funktionen eingeschlossen wird bertägt
$$A = \int_{a}^{b} ((f(x)-g(x)) \mathrm{d}x $$


\end{document}
