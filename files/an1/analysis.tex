\documentclass{article}
\usepackage{fancyhdr}
\pagestyle{fancy}

\author{Philipp Kiss}

\lhead{Philipp Kiss}
\rhead{AN1}

\begin{document}
\pagenumbering{gobble}

\section{Polynomfunktionen}
\subsection{Bestimmung von Nullstellen}
Wenn eine Nullstelle \(x_0\) einer Funktion \(y = f(x)\) vom Grad \textit{n} ist, dann gibt es eine eindeutige Polynomfunktion vom Grad \(n-1\)
\subsection{Ableitung}
Die Ableitung \(f'(x)\) der Funktion \(f(x)\) beschreibt die Steigung des Graphes von \(f(x)\). Die Ableitungsfunktion \(f'(x_0\) kann mit \[f'(x_0)= \lim_{h \to 0} \frac{(f(x_0 + h) - f(x_0)}{h} \]
berechnet werden. Man kann diesen Ausdruck vereinfachen zu \(f'(x_0) = nx^{n-1}\)

Aus einer komplexeren Polynomfunktion \[
		f(x) = 5x^4-2x^3+8x^2-x+17
\]
kann Päckchenweise die Funktion \[
		f'(x) = 20x^3-6x^2+16x-x^{-1}
\] abgeleitet werden.
\subsubsection{Graphisches Ableiten}
Um eine Funktion graphisch qualitativ abzuleiten muss man die lokalen Extrema, die Wendepunkte sowie die Steigung beachten. Bei den lokalen Extrema des Funktiongraphes sind die Nullstellen des Graphes der Ableitungsfunktion. Bei den Wendepunkten des ursprünglichen Graphes findet man wiederum die lokalen Extrema des abgeleiteten Graphes.
\subsection{Integral}
Mithilfe des Integrals kann man die Fläche zwischen einem Polynom und der X-Achse auf einem bestimmten Intervall berechnen.
\subsubsection{Stammfunktion}
Die Stammfunktion ist die Gegenoperation zur Ableitung und wird für eine Funktion \(f(x)\) mit \(F(x)\) beschrieben. Demnach gilt \(F'(x) = f(x)\)
\subsubsection{unbestimmtes Integral}
Ein unbestimmtes Integral wird mit der Notation
\[
		\int f(x)dx
\]
beschrieben. Dabei steht das \(f(x)\) für das Polynom. Das \(dx\) besagt auf welche Variable integriert wird.
\subsubsection{bestimmtes Integral}
\section{Folgen}
Elemente einer Folge heissen Glieder. Das Glied \(a_n\) ist das n-te Glied der Folge.
\subsection{Rekursiv definierte Folgen}
Rekursive Folgen geben in der Regel einen Startwert \(a_1\) vor und eine Formel für die Berechnung des nächsten Gliedes. Bsp: \[
a_1 = 2
\] \[
		a_n = a_{n-1} + a_{n-2}
\]
\subsection{Arithmetische Folgen}
Arithmetische Folgen zeichnen sich dadurch aus, dass die Differenz \(d\) zwischen den Gliedern immer gleich gross ist. Es gilt also \[
		a_n = \frac{a_{n-1} + a_{n+1}}{2} 
\]
\subsection{Geometrische Folgen}
Eine Geometrische Folge wird durch die Konstanz des Quotienten \(q\) bestimmt. Dabei gilt \[
		a_n = a_1 \cdot q^{n-1}
\]
Der Quotient \(q\) wird mit \[
		q = \frac{a_{k+1}}{a_k} 
\] berechnet.
\section{Reihen}
Eine Reihe bezeichnet die Summe aller Glieder einer Folge und wird mit einem grossen Sigma \(\Sigma\) notiert.
Dabei wird eine Laufvariable definiert, eine Obergrenze und einen Term. Bsp: \[
\sum_{i=0}^5 i = 0 + 1 + 2 + 3 + 4 + 5 = 15
\]
\subsection{Arithmetische Folgen}
Die Summe einer arithmetischen Folge wird wie folgt berechnet \[
		s_n = \sum_{i = 1}^n (a_1 + (i-1)\cdot d) = n \cdot a_1 + \frac{n(n-1)}{2} \cdot d
\]
\subsection{Geometrische Folgen}
Die Summe aller Glieder einer geometrischen Folge wird mithilfe von \[
		s_n = \sum_{i=1}^n a_1 \cdot q^{i-1} = a_1 \cdot \frac{q^n -1}{q-1} 
\] berechnet.

\subsection{unendliche Folgen}
\subsubsection{Geometrische Folgen}
\[
s_n = \frac{a_1}{1-q} 
\]
\section{Grenzwert}
Der Grenzwert, auch Limes genannt, ist der Wert, dem sich eine Funktion oder eine Folge zu einem bestimmten Punkt annähert. Notiert wird der Limes mit $$ \lim_{x \to y}$$wobei das $x$ für den Funktionswert und das $y$ für den anzunähernden Wert steht. Die Schwierigkeit beim vereinfachen von Termen mit Limes ist, dass man keine 0 im Zähler oder im Nenner hat.
\subsection{Existenz eines Grenzwertes}
Man kann überprüfen ob eine Funktion $f$ einen Grenzwert besitzt, sprich sie ist Konvergent, wenn es eine Zahl $a$ gibt für die gilt
$$ \lim_{n \to \infty}f(n) = a$$
Wenn eine Zahl $a$ in der Definitionsmenge vorhanden ist, ist $f$ Konvergent, ansonsten ist sie Divergent
\subsection{Standard-Grenzwerte}
$$ \lim_{n \to \infty} \frac{1}{n} = 0$$
$$ \lim_{n \to \infty} \sqrt[n]{a} = 1$$
$$ \lim_{n \to \infty} \left(1+ \frac{1}{n} \right)^{}n = e$$
\subsection{Übliche Lösungsmuster}
$$ \lim_{n \to \infty} \frac{2n^{6}-n^{3}}{7n^{6}+n^{5}-3} $$Lösung: erweitern mit $ \frac{1}{n^{k}}$, wobei $k$ der grösste Exponent ist.
$$ \lim_{n \to \infty} \frac{7^{n-1}+2^{n+1}}{7^{n}+5}$$
Lösung: erweitern mit $ \frac{1}{a^{k}} $, wobei $a$ die grösste Basis und $k$ der kleinste Exponent ist.
$$ \lim_{n \to \infty} \sqrt[]{a_n} - \sqrt[]{b_n}$$ Lösung: erweitern mit $ \sqrt[]{a_n}+ \sqrt[]{b_n}$
$$ \lim_{n \to \infty} \left( 1+ \frac{2}{3n} \right)^{4n}$$
Lösung: umformen zu $ \lim_{n \to \infty} \left(\left(1+ \frac{1}{x}\right)^{x}\right)^{a} = e^{a}$
\subsection{Stetige Funktionen}
Eine Funktion ist umgangssprachlich stetig, wenn der Funktionsgraph eine durchgehende Linie ist, sprich man kann ihn zeichnen, ohne den Stift abzusetzen.
Mathematisch definiert ist eine stetige Funktion $f$ mit einer Definitionsmenge $D$ definiert als
$$\forall_x \in D (f(x) = \lim_{n \to x}f(n))$$

\end{document}
