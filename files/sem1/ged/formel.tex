\documentclass[a4paper]{article}
\usepackage{blindtext}
\usepackage{geometry}
\geometry{
  a4paper,
  left=5mm,
  top=5mm}
\author{Philipp Kiss}
\begin{document}
\pagenumbering{gobble}
\section*{Elektrizität}
\begin{table}[h!]
		\begin{center}
				\begin{tabular}{|c|c|p{14.7cm}|}
						\hline
						\textbf{Abkürzung} & \textbf{Einheit} & \textbf{Beschreibung} \\
						\hline
						I & Ampère (A)& Die Stromstärke beschreibt den Stromfluss als Anzahl Elektronen pro Sekunde\\
						\hline
						U & Volt (V) & Die Spannung beschreibt den Druck, der durch die Abstossungskraft der Elektronen verursacht wird\\
						\hline
						R & Ohm (\(\Omega\)) & Der Widerstand beschreibt, wie stark ein Bauteil den Stromfluss behindert. \\
						\hline
						\(\rho\) & Ohm (\(\Omega\)) & Beschreibt den Widerstand eines spezifischen Materials\\
						\hline
						P&Watt (W) & Die Leistung die etwas braucht/generiert. \\
						\hline
						C&Farad (F)&Beschreibt die Kapazität die ein Kondensator speichern kann. \\
						\hline
						T&Sekunden (s)& Die Zeitkonstante beschreibt einen Schritt der Ladungsdauer eines Kondensators. \\
						\hline
						L&Henry (H) & Die Trägheit einer Spule. \\
						\hline
				\end{tabular}
		\end{center}
\end{table}

\subsection*{Kirchhoff}
\subsubsection*{Knotensatz} \(
		I_{ein} = I_{aus}
\)
\subsubsection*{Maschensatz} \(
U_0 = U_1 + ... + U_n
\)
\subsection*{Bauteile}
\subsubsection*{Widerstand}
\(
U=RI
\)
\subsubsection*{Kabelwiderstand}
\(
R= \rho \frac{l}{A} 
\)
\subsubsection*{Wirkungsgrad}
\(
		n = \frac{P_{output}}{P_{input}} 
\)
\subsubsection*{Kondensator}
\(
		\tau = e^{-1}
\)
\newline
\(
C= \frac{\tau}{R} 
\)
\newline 
\(
T = 2\pi \ \sqrt[]{LC}
\)
\subsubsection*{Spule}
\(
L= \frac{R \tau}{2} 
\)
\end{document}
